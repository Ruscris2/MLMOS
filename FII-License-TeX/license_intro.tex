\documentclass[a4paper,10pt]{article}
\usepackage[english]{babel}
\usepackage[utf8]{inputenc}
\usepackage{url}

\usepackage[nottoc]{tocbibind}


\title{Descriere licență - OpenPhoto}
\author{Rusu Cristian-Ștefan, 3B7}
\date{}

\begin{document}

\maketitle

\medskip

\large
\textbf{Context}\\
\normalsize

Încă de la inventarea aparatului foto, fiecare om a avut nevoia de a-și stoca într-un loc sigur și ușor accesibil cele mai frumoase 
amintiri obținute dealungul anilor. Acest lucru a fost făcut posibil, până în trecutul apropiat, de albumele foto. Acum în era digitală, 
standardul este de a avea în orice moment, în orice loc, și prin orice mediu acces la informații. 

\medskip
\large
\textbf{Motivație}\\
\normalsize

Acestă lucrare are ca scop realizarea unei platforme ușor de utilizat ce permite utilizatorilor să își găsească toate fotografiile într-un 
singur loc.

Platforma pune la dispoziție posibilitatea de a încărca fotografii de pe orice dispozitiv prin intermediul unei interfețe web. Orice 
fotografie încărcată pe platformă are posibilitatea de a avea o colecție de informații atașate acesteia, cum ar fi: nume, descriere,
 locație, persoane și/sau puncte de interes vizibile în fotografie. Aceste fotografii pot fi organizate în albume, pentru a le face mai 
ușor de gasit.

O altă funcționalitate ce stă la baza acestei platforme este editor-ul de fotografii. Odată încărcată o fotografie în sistem, aceasta 
poate fi deschisă în editor. Acesta lucrează cu conceptul de "nivel" (engl. layer). Orice fotografie ne-editată poate fi considerată un 
nivel. Sunt puse la dispoziție operații de baza asupra unui nivel, precum: translația (repoziționarea), scalarea, rotația, distorsionarea.
 Asupra unui nivel, deasemenea, se pot aplica diverse filtre de culoare. O altă opțiune de a obține un nivel, este crearea acestuia 
pornind de la o imagine transparentă. Asupra acesteia se pot aplica diverse unelte de desenare. În adiție se pot obține niveluri prin 
manipularea unor obiecte 3D primitive, oferite de către platformă, sau de încărcarea unui fișier 3D de către utilizator.

Toate fotografiile încărcate sau realizate în editor, pot fi făcute publice în diverse categorii, unde toți utilizatorii platformei le pot 
vizualiza, și pot oferi comentarii și rencenzii. 

\inputencoding{latin2}
\inputencoding{utf8}


\end{document}
